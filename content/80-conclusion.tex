%% ==============================
\chapter{Conclusion}
\label{sec:conclusion}
%% ==============================
This thesis seeks to answer the two research questions posed in the introduction. They are repeated here for better readability:

\begin{enumerate}
    \item How to navigate in complex multi-floor environments?
    \item How to plan paths that are straight and predictable?
\end{enumerate}

For both questions, a concept and an implementation have been described in this thesis. To solve navigation in a complex multi-floor environment, a hierarchical planner is proposed that uses an H-Graph to model arbitrary complex environments. In this graph, a search is performed to find the optimal solution path, which includes navigating between rooms, using elevators to different floors, or changing the building. This hierarchical planning algorithm can be extended to a theoretically unlimited number of hierarchy levels, depending on the needs of the environments. By providing a plugin for the common \gls{nav_2} stack, it can be used for any mobile robot using the ROS 2 framework. A behavior tree coordinates the actions required to perform a complete multi-floor navigation task. In this work, it was shown that the implementation of this concept works for a complex graph modeled from the \gls{iras} research campus as well as for a hospital simulation with two floors. Three mobile robots were used in the simulation to verify the interchangeability and possible application to a real robot. Finally, the system was partially implemented on the PeTRA robot. Due to proprietary drivers and problems with the ROS 1 bridge, the path planning was not done with the proposed planners, but with standard implementations. The complete behavior tree of the multi-floor navigation was successfully demonstrated on the real robot.

The second research question focuses on the PeTRA use case of multi-floor service tasks in hospitals. For this environment, coexistence with human workers is a major challenge. It is required that a mobile robot does not disturb the normal daily work in the area and at the same time follows a predictable behavior to avoid conflicts. These requirements are solved by the proposed straight path planner, which provides a roadmap of deterministic paths that are close to the walls and avoid disturbing large public spaces. The proposed ILIR planner uses the gridmap of a room to iteratively find the largest interior rectangles. These are combined into a polygon and connected to all bridge points such as doors or elevators. The resulting roadmap is then used to efficiently plan paths and, in combination with a collision avoidance controller, to safely navigate the mobile robot to its destination. By comparing the retrieved paths on a benchmark map with several common path planners such as A*, PRM, and RRT, it can be shown that the ILIR method plans paths that are generally closer to the walls, with less fluctuation, while being the fastest planner. The ILIR planner also outperforms the second best planner by up to 72 percentage-points on the newly introduced metric for measuring public space disturbance. Furthermore, a qualitative comparison with planners from related work in the field shows that the paths are straighter and more predictable for humans.

%% ==============================
\section{Contributions}
\label{sec:Contributions}
%% ==============================
The goal of this thesis is to answer the above research questions while at the same time filling a gap in the open source community of ROS 2. This thesis was only made possible due to the tremendous effort of people who make their work publicly available for free. The open source research community plays an important role in the increasingly fast development of robotics research and the democratization of knowledge. This thesis aims to contribute back to this ecosystem by making the implementation publicly available on GitHub (\url{https://github.com/AndreasZachariae/semantic_hierarchical_graph}). This work is licensed under a Creative Commons Attribution-NonCommercial 4.0 International License. The major contributions of this work are summarized in the list below:

\begin{itemize}
    \item A novel hierarchical planner for multi-floor navigation in arbitrarily large environments.
    \item A novel 2D planner called ILIR, capable of generating straight, deterministic, and predictable paths.
    \item A new metric to measure the disturbance of a path on shared human-robot spaces.
    \item A quantitative analysis of different planners with respect to the proposed metric.
    \item A qualitative comparison of ILIR's proposed roadmap with related work.
    \item A comprehensive evaluation of the proposed planner using realistic simulation scenarios.
    \item A proof-of-concept implementation of the planner on a real robot using \gls{ros_2} and \gls{bt}.
\end{itemize}

%% ==============================
\section{Future Work}
\label{sec:future_work}
%% ==============================
Future work on this topic will focus on improving the implementation to support dynamic map updates and efficient online replanning. This is one of the major limitations of the current system. Another improvement would be a more efficient hierarchical search that caches the precalculated path lengths of the connections in the subgraph at each hierarchy level. This would allow a more focused search with a heuristic like A*. The next important step would be to integrate this system on a full ROS 2 capable mobile robot like the MPO\_500. With this hardware it would be easy to demonstrate the full capability of this multi-floor navigation system in a real environment. Furthermore, a comprehensive benchmark environment for multi-floor scenarios is needed to test and compare methods with other researchers. This includes a generator that could randomly generate multi-floor models of complex buildings in simulation with their corresponding gridmaps. This could be used for two different steps: First, to develop a general algorithm to automatically generate the full H-Graph for a 4+ level environment. Currently, this is only done for two levels in related work. Second, once the H-Graph is built or provided, it could be used to compare different hierarchical planners. There is a lack of research on such a benchmark for evaluation. All of these advances could bring the field one step closer to easily deploying large fleets of robots in multi-floor buildings. Especially for service robots in hospitals, in the context of the increasing number of elderly people and the shortage of trained workers, the ability to perform a wide range of service tasks across multiple floors of an entire campus is a huge benefit to society as a whole.