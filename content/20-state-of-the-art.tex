%% ==============================
\chapter{State of the art}
\label{sec:state_of_the_art}
%% ==============================
Questions to ask:
\begin{enumerate}
    \item What is generally known about the topic? (relevant theory)
    \item What are the most important (recent) existing works in the field related to the research problem? 
    \item What are the limitations or gaps in the existing works that this work aims to address? 
    \item How does this work build upon or differ from existing works? 
    \item What are the main research approaches or techniques used in the field, and how do they relate to this work? 
    \item What are the main challenges or open questions in the field? 
\end{enumerate}

General theory:
\begin{enumerate}
    \item Path planning
    \item Astar, RRT, Dijkstra, PRM
    \item Hierarchical graphs
    \item ROS2 Nav Stack
    \item OpenRMF
\end{enumerate}

Other approaches to Q1 (Navigate in large multi-story environments):
\begin{enumerate}
    \item Paper IAS-17 Autonomous Hierachy Creation for Path Planning of Mobile Robots in Large Environments
    \item Paper Hierarchical Path-Planning for Mobile Robots Using a Skeletonization-Informed Rapidly Exploring Random Tree*
\end{enumerate}

Other approaches to Q2 (Plan paths that are straight, deterministic and human-predictable):
\begin{enumerate}
    \item Voronoi Diagrams, EVG (with sensory horizont)
    \item Smoothing algorithms (Bezier curves, B-Splines, Bechtold and Glavina, Latombe)
    \item Paper Straight Skeleton Based Automatic Generation of Hierarchical Topological Map in Indoor Environment
\end{enumerate}

Research gap from previous approaches:
\begin{enumerate}
    \item Hierarchical graphs are not per room or per floor which makes it difficult to append slamed maps or semantic information
    \item Hierarchical graphs are not used in ROS2 Nav Stack
    \item Approaches for straight paths are not deterministic and not human-predictable
    \item Approaches for straight paths are not implemented in ROS2 Nav Stack
    \item Approaches for straight paths only work well in small corridors
\end{enumerate}