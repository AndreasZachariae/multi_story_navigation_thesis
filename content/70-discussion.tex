%% ==============================
\chapter{Discussion}
\label{sec:discussion}
%% ==============================
Questions to ask:
\begin{enumerate}
    \item What effect/contribution does your result have?
    \item What limitations does your work have?
\end{enumerate}

\begin{enumerate}
    \item Limitations
    \item Not optimal paths
    \item Use-cases
\end{enumerate}

Additionally the proposed straight roadmap on the lowest level allows for efficient replanning. This is theoretically possible, but the process of live map updates and caching of room segmentations is not implemented. But the existing roadmap does not have to be raplanned all the time from the ground up and can be resued for all path queries.

In comparison, Seder \cite{seder_hierarchical_2011} uses the bridge points as hierarchy nodes in the graph. this way all possible paths can be easily precalculated and already on the highest hierachy level only the shortest path has to be explored deeper. This drawback of the current concept can be accounted for by two solutions: First by introducing a heuristic like euclidean distance which estimates the minimum distance between nodes on a lower level and thus enables it to search with a A* instead of dijsktras algorithm for the optimal path. The second solution would be to provide a mapping on the parent node which holds all the possible combinations of connections of possible start and goal nodes in the subgraph and its corresponding costs. This can be precalculated from bottom to top and then used later to efficiently search from top to bottom. Both of these solutions are currently not implemented. This concept uses the approach of exploring all possible solutions to its maximum depth and then comparing for the shortest solution.