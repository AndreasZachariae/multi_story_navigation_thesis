%% ==============================
\chapter{Discussion}
\label{sec:discussion}
%% ==============================
The results of this work show that the proposed system can solve the problem of multi-floor navigation. Furthermore, the proposed straight path planner produces a roadmap with straight, deterministic and predictable paths. In qualitative comparison with related work, the proposed ILIR planner has paths that are more straight, are in the center for corridors and avoid the disturbance of large public spaces. The limitation of the ILIR planner is its experimental implementation. Compared to the existing approaches of Cagigas \cite{cagigas_hierarchical_2005} and Seder et al. \cite{seder_hierarchical_2011}, the focus was not on efficient online replanning. This is theoretically possible, but the process of live map updates and caching of the largest rectangles is not implemented. However, once the existing roadmap is created, it can be immediately reused for multiple planning requests and does not need to be replanned from scratch. But as soon as a door or corridor is blocked, a complete recreation of the roadmap is required. This is due to the main assumption of this work, which is that the environment, and therefore the gridmap to plan on, is static and provides a complete representation of the planable space. 

The proposed hierarchical planner, on the other hand, is capable of efficiently replanning and reacting to a dynamic environment. Since the entire hierarchical structure of the environment is stored in the H-Graph, it is easy to remove or add certain nodes to the graph. For example, if the gridmap does not change, but the robot's start or goal positions in the room change, the new position is dynamically added to the existing roadmap without having to replan any part of the roadmap. This is done by straight line or A* search. Once this location is added to the graph of the corresponding room at the lowest hierarchical level, the hierarchical search for the shortest solution is very fast. Despite the short search time for a single path, all possible solutions must be compared to find the shortest path, which requires multiple H-graph searches. In comparison, Seder et al. \cite{seder_hierarchical_2011} uses the bridge points as hierarchy nodes in the graph. With this approach, all possible paths can be easily precomputed, and already at the highest hierarchy level, only the shortest path needs to be explored deeper. This drawback of the current approach can be addressed in two ways: First, by introducing a heuristic such as Euclidean distance, which estimates the minimum distance between nodes on a lower level and thus allows to search for the optimal path with an A* instead of Dijkstra's algorithm. The second solution would be to provide a mapping on the parent node that contains all possible combinations of connections of possible start and target nodes in the subgraph and their corresponding costs. This can be precomputed from bottom to top and then used later to efficiently search from top to bottom. Neither solution is currently implemented. This work uses the approach of exploring all possible solutions to their maximum depth and then comparing for the shortest solution.

The main contribution of this work is the combination of these two planners and their integration into the existing ROS 2 and Nav2 ecosystem. The last missing step is to apply this simulated concept to the real PeTRA robot. Normally, the gap between the simulation and the real world is very small. The problem here is that PeTRA uses an old version of the navigation stack implemented in ROS 1. This could easily be replaced with the \gls{nav_2} stack used in this work, but although the schedulers are the same, the motor drivers of the KUKA platform are also in ROS 1. These are proprietary drivers contained in inaccessible docker containers. This means that in order to apply this concept, a lot of work has to be done to make the hybrid system work with ROS 1 and 2. Despite the fact that in modern systems only ROS 2 should be used. The main problem with using both versions in parallel is the tightly integrated loop of the planner controller of the navigation with the feedback of the sensors for the current position. This is now split and redirected over the ROS 1 bridge, which is not as performant as the native solution. Due to time limitations of this work, it was not possible to fully integrate this system on the PeTRA robot, only some parts have been demonstrated.

In general, the contributions of this work advance the current state of the art in this research area by providing a comprehensive solution for multi-floor navigation integrated into ROS 2, which was previously unavailable in the community. Additionally, the previous work of \cite{ryu_hierarchical_2020} for room segmentation was extended and combined with the hierarchy generation of Gregoric et al. \cite{gregoric_autonomous_2022}. The H-Graph concept was extended to a general n-level implementation. Furthermore, the requirements of the hospital use case for straight and predictable paths were met and evaluated with a new metric to measure public space disturbance.