%% ==============================
\chapter{Discussion}
\label{sec:discussion}
%% ==============================
Questions to ask:
\begin{enumerate}
    \item What are the main implications and applications of the findings? 
    \item What are the main limitations and assumptions of the work? 
    \item What are the main strengths and weaknesses of the methodology, framework, and implementation? 
    \item How do the findings advance the state of the art in the field? 
    \item What are the main open questions and future directions of the work? 
\end{enumerate}

\begin{enumerate}
    \item Limitations
    \item Not optimal paths
    \item Use-cases
\end{enumerate}

The results of this work show that the proposed system can solve the problem of multi-floor navigation. Furthermore the proposed straight path planner produces a roadmap with straight, deterministic and human-predictable paths. In qualitative comparison with related work the proposed ILIR planner has paths that are more straight, are in the center for corridors and avoid the disturbance of large public spaces. The limitation of the ILIR planner is its experimental implementation. Compared to the existing approaches from \cite{cagigas_hierarchical_2005} and \cite{seder_hierarchical_2011} the focus was not on efficient online replanning. This is theoretically possible, but the process of live map updates and caching of the largest rectangles is not implemented. Although once the existing roadmap is created, it can be immediately reused for multiple planning requests. But as soon as some door or corridor is blocked it requires a full recreation of the roadmap. does not have to be replanned all the time from the ground up and can be reused for all path queries. This is due to the main assumption of this work, that the environment and therefor the gridmap to plan on is static and provides a complete representation of the planable space. 

The proposed hierarchical planner in comparison is capable of replanning efficiently and reacting to a dynamic environment. As the whole hierarchical structure of the environment is stored in the H-Graph, it is easy to remove or add certain nodes to the graph. For example if the gridmap does not change but the start or goal positions of the robot in the room change, the new position is dynamically added to the existing roadmap without the need for replanning any part of the roadmap. This is done by straight line connection or A* search. Once this location is added to the graph of the corresponding room on the lowest hierarchical level, the hierarchical search for the shortest solution is very quick. Despite the low search time for one single path, all possible solutions have to be compared to find the shortest path which requires multiple H-graph searches. In comparison, Seder \cite{seder_hierarchical_2011} uses the bridge points as hierarchy nodes in the graph. WIth this approach all possible paths can be easily precalculated and already on the highest hierarchy level only the shortest path has to be explored deeper. This drawback of the current concept can be accounted for by two solutions: First by introducing a heuristic like euclidean distance which estimates the minimum distance between nodes on a lower level and thus enables it to search with a A* instead of Dijkstra's algorithm for the optimal path. The second solution would be to provide a mapping on the parent node which holds all the possible combinations of connections of possible start and goal nodes in the subgraph and its corresponding costs. This can be precalculated from bottom to top and then used later to efficiently search from top to bottom. Both of these solutions are currently not implemented. This work uses the approach of exploring all possible solutions to its maximum depth and then comparing for the shortest solution.

The major contribution of this work is the combination of both of these planners and integrating them in the existing ROS 2 and Nav2 ecosystem. The final missing step is to apply this simulation-approved concept on the real PeTRA robot. Normally the gap from this simulation to the real world is very small. The problem here is that PeTRA uses an old version of the Navigation stack implemented in ROS 1. This could be easily switched with the \gls{nav_2} stack used in this work but although the planners are the same, the motor drivers of the KUKA platform are also in ROS 1. These are proprietary drivers contained in inaccessible docker containers. This means, that to apply this concept a lot of work has to be put into making the hybrid system with ROS 1 and 2 work together. Despite the fact that in modern systems only ROS 2 should be used. The main problem with using both version in parallel is the closely integrated loop of the planner controller of the navigation with the feedback of the sensors for the current location. This is now split and redirected over the ROS 1 bridge which is not as performant as the native solution. Due to time limitations of this work it was not possible to fully integrate this system on the PeTRA robot. However single components have been tested and work as planned with the current system. This is for example the whole structure of the a multi-floor planning request coordinated with the behavior tree. This was successfully tested and used for navigating in the lab with mock-ups for an potential elevator. Also a lot of work has be done by a student project to enable PeTRA to press elevator buttons with its integrated robot arm. This solves the problem of floor change where no wifi interface is available. With these components it was successfully demonstrated that the PeTRA system executes a multi-floor navigation task in a mock-up environment where the elevator was simulated by a cardboard box and the floor remains the same. The behavior tree presented in this work works exactly the same, the hierarchical planner can be used to plan to a goal while providing the intermediate step of driving to the elevator location. Just the straight path planner could not be integrated with ROS 1 as it would require a lot of parameter fine tuning to get the accompanying path controller to work as expected.

In general the contributions of this work advances the current state of the art in this research field by providing a comprehensive solution for multi-floor navigation integrated in ROS 2. Up to this point this was not available in the community. Additionally the previous work of \cite{ryu_hierarchical_2020} for room segmentation was extended and combined with the hierarchy creation of \cite{gregoric_autonomous_2022}. The concept of H-graph was extended to a general n-level implementation. Furthermore the requirements for the hospital use-case for straight and human-predictable paths were fulfilled and evaluated with a new metric to measure the disturbance of public space. 