% \newacronym{iras}{iRAS}{Institute for Robotics and Autonomous Systems}
\newacronym{ros}{ROS}{Robot Operating System}
\newacronym{ros_2}{ROS 2}{Robot Operating System 2}
\newacronym{petra}{PeTRA}{Personen-Transfer Roboter-Assistent}
\newacronym[plural=BTs,firstplural=Behavior Trees (BTs)]{bt}{BT}{Behavior Tree}
\newacronym[plural=FSMs,firstplural=Finite-State Machines (FSMs)]{fsm}{FSM}{Finite-State Machine}
\newacronym[plural=HFSMs,firstplural=Hierarchical Finite-State Machines (HFSMs)]{hfsm}{HFSM}{Hierarchical Finite-State Machine}
\newacronym{osrf}{OSRF}{Open Source Robotics Foundation}
\newacronym{slam}{SLAM}{Simultaneous Localization and Mapping}
\newacronym{uml}{UML}{Unified Modeling Language}
\newacronym{xml}{xml}{Extensible Markup Language}
\newacronym{lidar}{LIDAR}{Light Detection and Ranging}
\newacronym[plural=H-Graphs,firstplural=Hierarchical Graphs]{h_graph}{H-Graph}{Hierarchical Graph}

%% ==============================
\chapter{Introduction}
\label{sec:introduction}
%% ==============================
Questions to ask:
\begin{enumerate}
    \item What is your research topic? (From wide to narrow scope)
    \item What is the research problem or gap in the field that this work aims to address?
    \item Why is this problem or gap important?
    \item What are the research questions or hypotheses? 
    \item What is the scope of your work? (What are the limitations?)
    \item How is the following work structured?
\end{enumerate}

\begin{enumerate}
    \item General field of path planning for mobile robots
    \item Problem of path planning in multi-story environments
    \item Shared human-robot spaces
    \item Research question: How to plan in a multi-story environment? And how to drive in public spaces with service robots?
    \item Significance: Many use cases in public spaces in large hierarchical environments like hospitals.
    \item Development of a global planner plugin for straight and predictable paths in multi-story environments.
    \item Structure of the thesis
\end{enumerate}



Mobile robots are increasingly being used in large environments, such as hospitals, to improve efficiency and reduce human workload. However, navigating over multiple stories poses a significant challenge for mobile robots, as it requires a planner that can account for complex spatial configurations and still find the shortest path. Moreover, the planned path must be straight, deterministic and predictable for humans who could be walking in the same space as the robot. Therefore, the specific research problem addressed in this thesis is the development of a planner that can enable a mobile robot to navigate over multiple stories in large environments, while ensuring a straight, deterministic and human-predictable path. 

The ability to navigate over multiple stories is important for mobile robots in large environments, as it can significantly increase their utility and effectiveness. For instance, in hospitals, mobile robots could be used for tasks such as delivering medicines, transporting equipment, or guiding patients and visitors. Although for specific use-cases proprietary solutions exist, the lack of an open-source planner for multi-story navigation is a major bottleneck in the development of mobile robots for service applications. Moreover, the need for a straight, deterministic and human-predictable path is essential for ensuring safety and reducing the risk of collisions or accidents in shared human-robot spaces.

The two main research questions addressed in this thesis are:
\begin{enumerate}
    \item How to navigate in large multi-story environments?
    \item How to plan paths that are straight, deterministic and human-predictable?
\end{enumerate}

The scope of this work is limited to developing a planner which plans apriori with previously gathered information, in contrast to a controller which follows the planned path and adapts to dynamic obstacles. The main assumption is that the a map of the whole environment is previously recorded and provides a complete representation of the plannable space. The planner is implemented as a plugin for the \gls{ros_2} and is evaluated using realistic simulation scenarios and real-world experiments. The planner is designed to be modular and can be easily integrated with the existing navigation stack. The planner is also designed to be scalable and can be used in environments of varying sizes and complexity.

The main contributions of this thesis are:
\begin{itemize}
    \item A novel hierarchical planner for multi-story navigation in arbitrarily large environments
    \item A novel 2D planner which is able to generate straight, deterministic and human-predictable paths.
    \item A new metric to measure the disturbance of a path on shared human-robot spaces.
    \item A comparative analysis of different planners in respect to the proposed metric.
    \item A comprehensive evaluation of the proposed planner using realistic simulation scenarios.
    \item A proof-of-concept implementation of the planner on a real robot using \gls{ros_2} and \gls{bt}.
\end{itemize}

While the developed 2D planner is extensively evaluated in a simple 2D environment, the hierarchical planner is tested in a 3D simulation with a realistic environment. One example use-case for multi-story navigation are service applications in a hospital. This includes tasks such as delivering medicines and transporting a patient to their examiniations. This work is part of the \gls{petra} research project, which aims to develop a mobile robot for autonomously transporting patients in a hospital and completing other service tasks. As final proof-of-concept, the planner is implemented on the \gls{petra} robot and tested in a real-world hospital.

The following work is structured as follows: Chapter \ref{sec:state_of_the_art} provides an overview of the state of the art in the field of multi-story path planning for mobile robots. It also discusses the limitations of existing approaches. Chapter \ref{sec:methods} describes the methodology used to address the research questions. The concept of the planner which are \glspl{h_graph} is presented in Chapter \ref{sec:concept} followed by the implementation in Chapter \ref{sec:implementation}. This shows the approaches of creating a \gls{h_graph} from a previously recorded map. Chapter \ref{sec:results} presents the results of the work. Chapter \ref{sec:discussion} discusses the results and provides recommendations for future work. Chapter \ref{sec:conclusion} concludes the thesis.