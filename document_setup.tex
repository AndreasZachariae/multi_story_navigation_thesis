%% --------------
%% | Typography |
%% --------------

% T1 font encoding
\usepackage[T1]{fontenc}
\usepackage[utf8]{inputenc}
\usepackage[babel,german=guillemets]{csquotes}

%% --------------
%% | Packages   |
%% --------------
\usepackage[nooneline,bf]{caption} % Figure descriptions from left margin
\usepackage{times}

% numbers option provides compact numerical references in the text.
\usepackage[numbers]{natbib}

%% -------------------------------
%% | Graphics, Figures, Tables   |
%% -------------------------------
\usepackage[dvips,pdftex]{graphicx}
\usepackage{epic}
\usepackage{eepic}
\usepackage{epsfig}
\usepackage{tikz}
\usepackage{pgfplots} % necessary
\usepackage{float} % damit Tabellen und Grafiken sich selbst ausrichten dürfen und nicht zwangshaft an einer Stelle stehen. verhindern kann man das mit [H]
\usepackage{subcaption} % necessary
\usepackage{transparent}

\usepackage{multicol}
\usepackage{multirow} % necessary
\usepackage{longtable}
\usepackage[export]{adjustbox} % necessary
\usepackage{rotating}
\usepackage{hhline}
\usepackage{booktabs}

%% --------------
%% | Listings    |
%% --------------
\usepackage{algorithm}		  % Code-Listings
\usepackage{algpseudocode}    % algorithmicx with layout
%\usepackage{algorithmic}	  % Code-Listings
% see http://www.ctan.org/tex-archive/macros/latex/contrib/algorithm2e/algorithm2e.pdf
% for more sophisticated algorithm listings
\usepackage{listings}

\definecolor{darkGray}{RGB}{135,135,134}
\definecolor{CodeGreen}{rgb}{0,0.6,0}
\definecolor{CodeMauve}{rgb}{0.58,0,0.82}
\lstset{captionpos=t, numbers=left, stepnumber=1, tabsize=2, basicstyle=\ttfamily\footnotesize, breaklines=true, xleftmargin=4em, frame=leftline, rulecolor=\color{darkGray}, framexleftmargin=2em, escapechar=\%, numberstyle=\color{darkGray},commentstyle=\color{CodeGreen}, keywordstyle=\color{blue}, stringstyle=\color{CodeMauve}}
% Make listings similar to algorithm environment
% \DeclareCaptionFormat{mylst}{\hrule#1#2#3}
% \captionsetup[lstlisting]{format=mylst,labelfont=bf,singlelinecheck=off,labelsep=space}

%% ------------------------------------------
%% | Glossary, Acronyms and Nomenclature    |
%% ------------------------------------------
\usepackage[acronym,toc,nonumberlist]{glossaries}
\usepackage{nomencl}

\makeglossaries

%% --------------
%% | Margins    |
%% --------------
\usepackage{vmargin}          % Adjust margins in a simple way
\usepackage[absolute,overlay]{textpos}

%% --------------
%% | Math       |
%% --------------
\usepackage{grffile}
\usepackage{amsmath}
\usepackage{amstext}
\usepackage{amssymb}
\usepackage{bm}
\usepackage{spverbatim}
\usepackage{textcomp}
\usepackage{enumerate}
\usepackage[inline]{enumitem}

%% ------------------------------
%% | TODOs and Dummy Text       |
%% ------------------------------
\iflanguage{ngerman}{
  \usepackage[colorinlistoftodos,ngerman]{todonotes} %See: https://www.ra.informatik.tu-darmstadt.de/fileadmin/user_upload/Group_RA/eiwa/todonotes.pdf
}{
  \usepackage[colorinlistoftodos]{todonotes} %See: https://www.ra.informatik.tu-darmstadt.de/fileadmin/user_upload/Group_RA/eiwa/todonotes.pdf
}

\usepackage{blindtext}

%% ---------------------------
%% | Line spacing            |
%% ---------------------------
\usepackage{setspace}
\newcommand{\MSonehalfspacing}{%
    \setstretch{1.44}%  default
    \ifcase \@ptsize \relax % 10pt
        \setstretch {1.448}%
    \or % 11pt    
        \setstretch {1.399}%
    \or % 12pt
        \setstretch {1.433}%
    \fi
}
\newcommand{\MSdoublespacing}{%
    \setstretch {1.92}%  default
    \ifcase \@ptsize \relax % 10pt
        \setstretch {1.936}%
    \or % 11pt
        \setstretch {1.866}%  
    \or % 12pt
        \setstretch {1.902}%
    \fi
}

%% ---------------------------
%% | References in PDF       |
%% ---------------------------
% Hyperref should be the last package loaded
\usepackage[hyphens]{url}
\usepackage[breaklinks,colorlinks=true,pdftex,pageanchor=false]{hyperref}
\Urlmuskip=0mu  plus 10mu

%% -------------------------------
%% |        Declarations         |
%% -------------------------------
\pgfplotsset{compat=newest}
\pgfplotsset{plot coordinates/math parser=false}
%% the following commands are needed for some matlab2tikz features
\usetikzlibrary{plotmarks}
\usetikzlibrary{arrows.meta}
\usepgfplotslibrary{patchplots}

\graphicspath{{figures/}{../jpeg/png/svg/gif/pdf/}}

\setlength {\marginparwidth }{2cm}

\newcommand\setpdf{

  \let\theauthor\@author
  \let\thetitle\@title

  \hypersetup{
    plainpages=false,
    linkcolor=black,          % color of internal links (change box color with linkbordercolor)
    citecolor=black,        % color of links to bibliography
    filecolor=black,      % color of file links
    urlcolor=black           % color of external links
  }
}

\newcommand\tab[1][1cm]{\hspace*{#1}}

\newcommand{\comment}[1]{}

%% --- End of Declarations ---

